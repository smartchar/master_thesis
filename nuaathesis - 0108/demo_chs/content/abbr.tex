% 注释表和缩略词,硕博论文用。
% 《要求》没有规定内容格式,按照自己的喜好来改吧。
% 注意,表格里的文字不要太长哦。

\chapter*{注释表}

\noindent\begin{longtabu} to \textwidth {|X[c]|p{5.0cm}|X[c]|p{5.0cm}|}\hline
\multirow{2}{*}{$x_{sA}$} & \multirow{2}{5.0cm}{转子几何轴在A端位移传感器截面处$x$轴方向的位移} &
$m_A $ & 校正盘A上的增重质量大小 \\ \cline{3-4}
& & $m_B $ & 校正盘B上的增重质量大小 \\ \hline

\multirow{2}{*}{$y_{sA}$} & \multirow{2}{5.0cm}{转子几何轴在A端位移传感器截面处$y$轴方向的位移} &
$\phi _A$ & 校正盘A上的增重质量相位 \\ \cline{3-4}
& & $\phi _B$ & 校正盘B上的增重质量相位 \\ \hline

\multirow{2}{*}{$x_{sB}$} & \multirow{2}{5.0cm}{转子几何轴在B端位移传感器截面处$x$轴方向的位移} &
$k_f$ & 陷波器增益 \\ \cline{3-4}
& & $\theta _f$ & 陷波器相位角 \\ \hline

\multirow{2}{*}{$y_{sB}$} & \multirow{2}{5.0cm}{转子几何轴在B端位移传感器截面处$y$轴方向的位移} &
$\Omega _f$ & 陷波器中心频率 \\ \cline{3-4}
& & $N_f(s)$ & 陷波器开环传递函数 \\ \hline

\multirow{2}{*}{$x_{bA}$} & \multirow{2}{5.0cm}{转子几何轴在A端磁轴承处$x$轴方向的位移} &
$N(s)$ & 陷波器闭环传递函数 \\ \cline{3-4}
& & $S_0(s)$ & 输出敏感度函数 \\ \hline

\multirow{2}{*}{$y_{bA}$} & \multirow{2}{5.0cm}{转子几何轴在A端磁轴承处$y$轴方向的位移} &
$\varepsilon $ & 重复控制器控制增益 \\ \cline{3-4}
& & $f_s$ & 控制频率 \\ \hline

\multirow{2}{*}{$x_{bB}$} & \multirow{2}{5.0cm}{转子几何轴在B端磁轴承处$x$轴方向的位移} &
$T_s$ & 控制周期 \\ \cline{3-4}
& & $k_{sA}$ & A端磁轴承位移刚度 \\ \hline

\multirow{2}{*}{$y_{bB}$} & \multirow{2}{5.0cm}{转子几何轴在B端磁轴承处$y$轴方向的位移} &
$k_{sB}$ & B端磁轴承位移刚度 \\ \cline{3-4}
& & $k_{iA}$ & A端磁轴承电流刚度 \\ \hline

\multirow{2}{*}{$I_x$、$I_y$、$I_z$} & \multirow{2}{5.0cm}{转子绕$x$、$y$和$z$轴的惯性力矩} &
$k_{iB}$ & B端磁轴承电流刚度 \\ \cline{3-4}
& & $k_s$ & 位移刚度 \\ \hline

\multirow{2}{*}{$\boldsymbol{q _s}$} & \multirow{2}{5.0cm}{位移传感器坐标下转子几何轴位移} &
$f_c $ & 截止频率 \\ \cline{3-4}
& & $\omega _s$ & 控制角频率 \\ \hline

\multirow{2}{*}{$\boldsymbol{q_{\Delta}}$} & \multirow{2}{5.0cm}{转子几何轴与惯性轴之间的偏移量} &
$\omega _d$ & 扰动信号基波角频率 \\ \cline{3-4}
& & $k_s$ & 位移刚度 \\ \hline

\multirow{2}{*}{$\boldsymbol{{\hat q}_s}$} & \multirow{2}{5.0cm}{位移传感器坐标下转子几何轴观测位移} &
$f_m$ & 磁悬浮力 \\ \cline{3-4}
& & $k_i$ & 电流刚度 \\ \hline

\multirow{2}{*}{$\boldsymbol{z}_c(t) $} & \multirow{2}{5.0cm}{转子端面的质量中心的平面轨迹} &
$\boldsymbol{z}_g(t) $ & 转子端面几何中心的平面轨迹 \\ \cline{3-4}
& & $\boldsymbol{z}_{\Delta}(t) $ & 转子端面偏心距的平面轨迹 \\ \hline

\multirow{2}{*}{$\boldsymbol{f}_z(t) $} & \multirow{2}{5.0cm}{转子端面的磁悬浮力的平面轨迹} &
$G_w(s)$ & 功率放大器传递函数 \\ \cline{3-4}
& & $G_c(s)$ & PID控制器传递函数 \\ \hline

\multirow{2}{*}{$\boldsymbol{\eta} _g(t)$} & \multirow{2}{5.0cm}{广义坐标系下转子几何轴中心的平动位移} &
$G_s(s)$ & 位移传感器传递函数 \\ \cline{3-4}
& & $G_{rc}(z)$ & 传统重复控制器开关传递函数 \\ \hline

\multirow{2}{*}{$\boldsymbol{\nu} _g(t)$} & \multirow{2}{5.0cm}{广义坐标系下转子几何轴中心的转动位移} &
$l_{bA}$ & A端磁轴承到质心的距离 \\ \cline{3-4}
& & $l_{bB}$ & B端磁轴承到质心的距离 \\ \hline

\multirow{2}{*}{$\boldsymbol{\eta} _{\Delta}(t)$} & \multirow{2}{5.0cm}{广义坐标系下转子偏心距离的平动位移} &
$l_{sA}$ & A端位移传感器到质心的距离 \\ \cline{3-4}
& & $l_{sB}$ & B端位移传感器到质心的距离 \\ \hline

\multirow{2}{*}{$\boldsymbol{\nu} _{\Delta}(t)$} & \multirow{2}{5.0cm}{广义坐标系下转子偏心距离的转动位移} &
$l_{cA}$ & A端校正盘到质心的距离 \\ \cline{3-4}
& & $l_{cB}$ & B端校正盘到质心的距离 \\ \hline

$\boldsymbol{q_i}$ & 广义坐标下转子惯性轴位移 & $\boldsymbol{q_b}$ & 轴承坐标下转子几何轴位移 \\ \hline
$\boldsymbol{q _g}$ & 广义坐标下转子几何轴位移 & $\Theta {\boldsymbol{q}_s}$ & 位移传感器测量误差 \\ \hline
$\beta _i$ & 转子惯性轴绕$y$轴的旋转角度 & $\beta _g$ & 转子几何轴绕$y$轴的旋转角度 \\ \hline
$\alpha _i$ & 转子惯性轴绕$x$轴的旋转角度 & $\alpha _g$ & 转子几何轴绕$x$轴的旋转角度 \\ \hline
$x_i$ & 转子惯性轴在$x$轴方向的位移 & $x_g$ & 转子几何轴在$x$轴方向的位移 \\ \hline
$y_i$ & 转子惯性轴在$x$轴方向的位移 & $y_g$ & 转子几何轴在$x$轴方向的位移 \\ \hline
$\sigma$ & 转子动平衡的幅值 & $\gamma$ & 转子动不平衡的初相位 \\ \hline
$\xi$ & 转子静平衡的幅值 & $\theta$ & 转子静不平衡的初相位 \\ \hline
$i_{XA}$ & XA通道控制电流 & $i_{XB}$ & XB通道控制电流 \\ \hline
$i_{YA}$ & YA通道控制电流 & $i_{YB}$ & YB通道控制电流 \\ \hline





\end{longtabu}

\chapter*{缩略词}

\noindent\begin{tabu} to \textwidth {|X[1,c]|X[4,c]|}\hline
缩略词 & 英文全称 \\ \hline
CRC & Conventional Repetitive Controller \\ \hline
ZORC & Zero-phase Odd-harmonic Repetitive Controller \\ \hline
PMB & Permanent Magnetic Bearing \\ \hline
HMB & Hybrid Magnetic Bearing \\ \hline
ISMB & International Symposium on Magnetic Bearings \\ \hline
CSMB & Chinese Symposium on Magnetic Bearings \\ \hline
LMS & Least Mean Square \\ \hline
PID & Proportion Integration Differentation\\ \hline
DSP & Digital Signal Processor \\ \hline
ADC & Analog to Digital Converter\\ \hline
FPU & Floating Point Uint\\ \hline
FPGA & Field Programmable Gate Array\\ \hline
PLL & Phase Locked Loop\\ \hline
\end{tabu}
